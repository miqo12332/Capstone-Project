\documentclass[12pt]{article}

% ===========================
% Packages
% ===========================
\usepackage[margin=1in]{geometry}
\usepackage{hyperref}
\usepackage{graphicx}
\usepackage{array}
\usepackage{longtable}
\usepackage{enumitem}
\usepackage{titlesec}
\usepackage{fancyhdr}
\usepackage{amsmath}
\usepackage{float}

% ===========================
% Formatting
% ===========================
\titleformat{\section}{\large\bfseries}{\thesection}{0.75em}{}
\titleformat{\subsection}{\normalsize\bfseries}{\thesubsection}{0.75em}{}
\titleformat{\subsubsection}{\normalsize\itshape}{\thesubsubsection}{0.75em}{}

\pagestyle{fancy}
\fancyhf{}
\rhead{StepHabit Capstone}
\lhead{Comprehensive Report}
\cfoot{\thepage}

\begin{document}

% ===========================
% Title Page
% ===========================
\begin{titlepage}
  \centering

  % Logo (reduced size)
  \includegraphics[width=0.55\textwidth]{images/Aua_logo.png}\par
  \vspace{1cm}

  % Course
  {\Large \textbf{CSE 296: Capstone Project}\par}
  \vspace{0.8cm}

  % Project Name
  {\Huge \textbf{StepHabit}\par}
  \vspace{0.6cm}

  % Subtitle
  {\large AI-Driven Intelligent Scheduling and Micro-Progression for\\
  Long-Term Habit Formation\par}
  \vspace{1cm}

  % University
  {\Large American University of Armenia\par}
  \vspace{1.2cm}

  % Supervisor and Program Chair
  \begin{flushleft}
    \centering
    \normalsize
    \textbf{Supervisor:} Aleksandr Hayrapetyan\\[0.4cm]
    \textbf{Group Members:} Mikayel Davtyan, Artur Aghamyan\\[0.4cm]
    \textbf{Program Chair:} Hayk Nersisyan
  \end{flushleft}

  \vfill

  % Location and Date
  {\normalsize Yerevan, Armenia\par}
  {\normalsize December 17, 2025\par}

\end{titlepage}

\clearpage

\begin{abstract}
  
  
  StepHabit is an AI-powered platform designed to help individuals build meaningful and lasting habits in a way that feels natural and sustainable. Unlike traditional habit trackers that rely on streaks or large commitments, StepHabit emphasizes micro steps that gradually evolve into stronger routines. By focusing on consistency before intensity, the platform ensures that small actions become deeply rooted behaviors over time. The system gives opportunity to create habits, tasks, to schedule them, to get friends and participate in group challenges. One of major components in platform is HabitCoach, which is AI-driven assistant that will help user with all the steps, will remember information provided by user, tell about statistics, help to improve user's progress, create notifications to remind about essential events. This report presents a detailed technical and product-level analysis of StepHabit, showing system architecture, data modeling, providing info about both backend and frontend services and about implementation decisions. The document is the reference for all users and for programmers who want either to use StepHabit for their progress or want to have their contribution for future improvement.

  
\end{abstract}

\newpage
\tableofcontents
\newpage

% ===========================
% Chapter 1
% ===========================
\section{Introduction}
\subsection{System Overview}

StepHabit is a secure, AI-assisted platform. It is created to help
users transform long-term goals into consistent daily routines. The system demonstrates strong focus on the structure, motivation, and sustainability. It allows users to manage, create, and track tasks, habits and schedules, even to use ready templates for habit creation A secure registration and authentication process ensures account
integrity. At the same time personalized dashboards allow users to monitor progress, streaks and achievements over time.

Once authenticated, users can define habits with daily goals, manage tasks with durations and deadlines, and organize their time through calendar-based planning and busy-block scheduling. The platform provides visual feedback through graphs, charts, and progress logs, helping users clearly understand how their daily actions influence reaching goals. Notifications, reminders make users more engaged and more responsible in achieving expected results. Besides traditional productivity tools, platform integrates artificial intelligence to
provide personalized coaching, statistics and guidance. An AI-driven assistant or HabitCoach as it is called in application, analyzes user
input and context to generate structured habit plans, user-related habit ideas, and offer actionable suggestions. By maintaining assistant memory tied to individual users, the system delivers increasingly relevant feedback.

User data is stored securely and is accessible only to authenticated users, with all core features scoped to individual accounts to ensure privacy. Social features such as friends, messaging, and group challenges are implemented within controlled
boundaries, allowing users to engage with each other without
compromising personal data. Calendar integrations help users to see what events they should have each day. HabitCoach will not let you have several events at the same time if you try to add, but if a user wants he/she can add several events at the same time either expecting to do several actions during some period or put it just on schedule and decide what to remove or edit later.

In summary, StepHabit combines secure account management, structured planning and AI-powered coaching to deliver a comprehensive and
intelligent productivity experience. The platform demonstrates how modern web technologies and artificial intelligence can be wisely integrated to support sustainable habit formation and long-term personal growth.

\subsection{Objectives}
The primary objective of this project is to design and develop a secure, AI-assisted
productivity and habit-building platform that enables users to translate long-term
goals into consistent daily actions. The system aims to provide a structured yet
flexible environment in which users can create, manage, and track habits, tasks, and
schedules, while receiving intelligent guidance to improve planning and execution.
Through AI-powered coaching, the platform supports users in refining habit ideas,
breaking goals into actionable steps, and maintaining motivation over time.

In addition, the project seeks to deliver an intuitive and user-friendly interface,
secure account management through authenticated access, and personalized dashboards
that visualize progress, streaks, and achievements. By combining scheduling tools,
progress tracking, and social accountability mechanisms, StepHabit encourages
sustainable behavior change rather than short-term productivity bursts.

Secondary objectives include building a scalable full-stack web application using
modern frameworks, ensuring data security and user privacy, and establishing a
robust architectural foundation that can support future extensions such as mobile
clients, advanced analytics, or enhanced AI-driven personalization.
\subsection{Scope}
The primary objective of this project is to design and develop a secure, AI-assisted
productivity and habit-building platform that enables users to translate long-term
goals into consistent daily actions. The system aims to provide a structured yet
flexible environment in which users can create, manage, and track habits, tasks, and
schedules, while receiving intelligent guidance to improve planning and execution.
Through AI-powered coaching, the platform supports users in refining habit ideas,
breaking goals into actionable steps, and maintaining motivation over time.

In addition, the project seeks to deliver an intuitive and user-friendly interface,
secure account management through authenticated access, and personalized dashboards
that visualize progress, streaks, and achievements. By combining scheduling tools,
progress tracking, and social accountability mechanisms, StepHabit encourages
sustainable behavior change rather than short-term productivity bursts.

Secondary objectives include building a scalable full-stack web application using
modern frameworks, ensuring data security and user privacy, and establishing a
robust architectural foundation that can support future extensions such as mobile
clients, advanced analytics, or enhanced AI-driven personalization.


\newpage
\section{Methodology}

During the initial phases of the project, secondary research was conducted to support the
development of the product. The research involved peer-reviewed academic literature, industry
reports, and existing digital habit-tracking applications. The findings helped to establish the need
for the system, identify prevailing design patterns, and inform the definition of system
requirements.

\subsection{Justification for the Need of the Product}

Findings from prior studies and industry analysis show that individuals often struggle to
maintain consistent habits due to a lack of motivation, competing responsibilities, and the
absence of structured feedback. Secondary research further indicates that, despite the wide
availability of habit-tracking applications, sustained user engagement remains limited.

Prior research by Lally et al.\ (2010)\footnote{Lally, P., van Jaarsveld, C. H. M., Potts, H. W. W., \& Wardle, J. (2010). How are habits formed: Modelling habit formation in the real world. \textit{European Journal of Social Psychology, 40}(6), 998--1009. https://doi.org/10.1002/ejsp.674}
examined the process of habit formation in real-world settings and demonstrated that habit
development occurs gradually and varies substantially between individuals. The study found
that repeatedly performing a behavior in a consistent context supports the development of
automaticity, while missing occasional instances does not necessarily disrupt the habit
formation process. These findings are directly relevant to the design of StepHabit, as they
support long-term tracking, consistency-oriented reminders, and tolerance for missed
instances, rather than an assumption of short-term behavior change.

The study by Stawarz, Cox, and Blandford (2015)\footnote{Stawarz, K., Cox, A. L., \& Blandford, A. (2015). Beyond self-tracking and reminders: Designing smartphone apps that support habit formation. \textit{Proceedings of the 33rd Annual ACM Conference on Human Factors in Computing Systems}, 2653--2662. https://doi.org/10.1145/2702123.2702230}
provides one of the most comprehensive academic examinations of how applications support
habit formation. Through both an empirical study and a large-scale review of existing habit-
tracking systems, the authors show that most mobile applications rely primarily on self-tracking
and time-based reminders, while offering limited support for contextual cues and implementation
intentions. Although reminders may improve short-term adherence, the findings indicate that
they may hinder the development of automaticity by encouraging reliance on technology rather
than fostering habitual behavior. These insights inform the design of StepHabit, which applies
habit-based design principles without claiming guaranteed behavioral change outcomes.

\subsection{Market Research}

An examination of widely used habit-tracking applications reveals several established systems
that support basic habit management functionality. Commonly used applications include
Streaks, Habitica, Habitify, Loop Habit Tracker, and Fabulous. For example, Streaks emphasizes
maintaining completion streaks and provides visual feedback on daily progress\footnote{Streaks. (n.d.). \textit{Streaks: Habit tracking app}. https://streaksapp.com}, while Habitica motivates users through gamification
elements such as points and virtual rewards\footnote{Habitica. (n.d.). \textit{Habitica: Gamified habit-building and productivity app}. https://habitica.com}.
Habitify focuses on habit tracking complemented by basic analytics and performance
summaries\footnote{Habitify. (n.d.). \textit{Habitify: Habit tracker and routine planner}. https://habitify.me}, whereas Loop Habit Tracker adopts a minimalist,
open-source approach centered on manual tracking with limited personalization\footnote{Soron, A. (n.d.). \textit{Loop Habit Tracker} [Open-source software]. GitHub. https://github.com/iSoron/uhabits}.

Although these applications effectively support self-monitoring and short-term engagement,
academic and industry analyses indicate that they largely rely on static reminders and
descriptive feedback, offering limited adaptive or context-aware support for sustained habit
development\footnote{Zapier. (2025). \textit{The best habit tracker apps}. https://zapier.com/blog/best-habit-tracker-app/}.

While existing habit-tracking applications provide valuable baseline functionality, StepHabit is
designed to address several limitations identified through prior research and market analysis.
Most current systems provide descriptive feedback—such as streak counts or completion
statistics—which inform users of what has occurred but offer limited insight into why
performance changes or how habits might be adjusted.

In contrast, StepHabit integrates AI-assisted feedback mechanisms intended to interpret user
behavior patterns and provide contextual, non-punitive guidance. Rather than focusing solely on
streak maintenance, the platform emphasizes reflective feedback and adaptive suggestions to
support sustained engagement. These AI-driven features are positioned as decision-support
tools that enhance personalization and usability rather than mechanisms that guarantee
behavioral change. This approach aligns with academic findings emphasizing the importance of
contextual understanding and long-term engagement over reliance on repetitive notifications.

\newpage
% ===========================
% Chapter 2
% ===========================
\section{System Architecture}
\subsection{Overview of System Components}
The StepHabit platform is designed using a modular system architecture, in which each
major component is responsible for a clearly defined set of responsibilities. This
architectural approach improves clarity, maintainability, and scalability, while
allowing individual components to evolve independently. Together, these components
form a cohesive, secure, and intelligent productivity system that supports habit
formation, task management, and goal-oriented behavior.

The frontend application serves as the primary interaction layer between users and the
system. It provides interfaces for user registration and authentication, habit and task
creation, calendar-based planning, and progress tracking. Through dashboards,
planners, and visual analytics, users can monitor habits, streaks, achievements, and
upcoming tasks. Social features such as messaging, group challenges, and notifications
are also accessible through the frontend, enabling accountability-driven engagement in
a user-friendly and responsive interface.

The backend server implements the core business logic and coordinates communication
between the frontend, the database, and external services. It manages authentication
and authorization, enforces data ownership and privacy rules, and processes all user
actions, including habit updates, task scheduling, progress logging, notifications, and
community interactions. The backend also exposes RESTful APIs that provide a stable
and extensible interface for current and future clients.

The database layer is responsible for persistent data storage and integrity. It stores
user profiles, habits, tasks, schedules, progress records, achievements, notifications,
assistant memory, and social relationships. The relational schema is designed to support
clear ownership boundaries between users and their data, while enabling efficient
queries for analytics, dashboards, and planner views.

An AI-powered coaching component enhances the platform by providing intelligent
guidance and personalization. This service analyzes user-provided habit ideas and
contextual data to generate structured habit plans, refine goal descriptions, and offer
actionable suggestions. Assistant memory is maintained on a per-user basis, allowing
the system to deliver increasingly relevant and consistent coaching over time. AI
outputs are evaluated and stored where appropriate to support reflection and future
interactions.

Together, these components form a robust and extensible system that integrates
structured planning, progress visibility, social accountability, and AI-driven guidance.
StepHabit combines modern web technologies with intelligent feedback mechanisms to
deliver a productivity platform that supports sustainable habit formation and long-term
personal growth.
\subsection{Technology Stack and Frameworks}
The StepHabit platform utilizes a modern, modular technology stack designed to ensure
performance, scalability, maintainability, and ease of future extension across all
layers of the system. Each technology was selected to support rapid development while
maintaining architectural clarity and long-term sustainability.

\subsubsection{Frontend}
\begin{itemize}[leftmargin=*]
  \item \textbf{React with Vite}: Builds responsive and interactive user interfaces while
  leveraging Vite for fast development cycles and optimized production bundles.
  \item \textbf{CoreUI}: Provides component-based layouts, navigation structures,
  dashboards, and visualization patterns used across the application.
\end{itemize}

\subsubsection{Backend}
\begin{itemize}[leftmargin=*]
  \item \textbf{Node.js with Express}: Supplies the runtime and HTTP framework used for
  routing, middleware composition, and request handling.
  \item \textbf{Sequelize ORM}: Manages relational interactions with PostgreSQL through
  model definitions, associations, and schema synchronization.
  \item \textbf{JWT (JSON Web Tokens)}: Secures authentication and authorization so
  protected API endpoints remain accessible only to authenticated users.
  \item \textbf{Bcrypt}: Protects credentials through password hashing.
  \item \textbf{Socket-Based Messaging (Planned)}: Supports future integration of
  WebSocket-powered notifications and messaging.
\end{itemize}

\subsubsection{Database}
\begin{itemize}[leftmargin=*]
  \item \textbf{PostgreSQL}: Stores structured data for profiles, habits, tasks,
  schedules, progress logs, achievements, notifications, and social relationships while
  preserving integrity for analytics and dashboard queries.
\end{itemize}

\subsubsection{AI Integration}
\begin{itemize}[leftmargin=*]
  \item \textbf{LangChain with Large Language Models (Anthropic)}: Powers AI-assisted
  habit coaching, plan generation, idea refinement, and contextual guidance through a
  flexible abstraction layer for prompt management and provider configuration.
\end{itemize}

\subsubsection{Configuration Management}
\begin{itemize}[leftmargin=*]
  \item \textbf{Environment-Based Configuration}: Uses environment files to secure
  sensitive credentials and adapt deployments across development and production
  environments.
\end{itemize}
\subsection{Development and DevOps Tools}
The development of the StepHabit platform relied on a set of modern development and
DevOps tools to support efficient implementation, testing, collaboration, and
deployment. These tools contributed to code quality, system reliability, and
developer productivity throughout the project lifecycle.

\begin{itemize}[leftmargin=*]
  \item \textbf{Visual Studio Code (VS Code)}: Primary IDE for frontend and backend
  development, providing language support, debugging, and extension-based tooling for
  full-stack workflows.
  \item \textbf{Postman}: Used to test RESTful API endpoints, validate payloads, and
  debug authentication and business logic.
  \item \textbf{Git and GitHub}: Supports version control, repository hosting, issue
  tracking, and pull-request-driven collaboration.
  \item \textbf{Docker and Docker Desktop}: Containerizes backend services, the frontend
  client, and the database; Docker Compose orchestrates multi-container environments.
  \item \textbf{Navicat}: Provides database administration, table inspection, and data
  validation for PostgreSQL during development and testing.
  \item \textbf{ESLint and Prettier}: Enforce coding standards, detect issues early, and
  maintain readability across the codebase.
  \item \textbf{Email and Notification Monitoring Tools}: Verify account verification and
  notification delivery workflows for user-facing communication.

\end{itemize}


\newpage

% ===========================
% Chapter 3
% ===========================
\section{Database Design}
\subsection{Database Choice}
PostgreSQL was chosen for this project because of its excellent support for relational data,
adherence to ACID principles, and support for TypeORM integration. PostgreSQL's support for
complex querying and indexing provides an effective solution to managing the complex
relationships between users, posts, and interactions (likes, comments).
\subsection{Entity Relationship Diagram (ERD)}
ERD 
\subsection{Database Tables}
\small

\textbf{Users Table}
\begin{itemize}
  \item \textbf{Table Name:} users
  \item \textbf{Fields:}
  \begin{itemize}
    \item id (SERIAL, Primary Key)
    \item name (VARCHAR(100), Not Null)
    \item email (VARCHAR(150), Unique, Not Null)
    \item password (VARCHAR(200), Not Null)
    \item age (INT, Nullable)
    \item gender (VARCHAR(20), Nullable)
    \item bio (TEXT, Nullable)
    \item avatar (VARCHAR(255), Nullable)
    \item primary\_goal (VARCHAR(150), Nullable)
    \item focus\_area (VARCHAR(120), Nullable)
    \item experience\_level (VARCHAR(60), Nullable)
    \item daily\_commitment (VARCHAR(60), Nullable)
    \item support\_preference (VARCHAR(120), Nullable)
    \item motivation\_statement (TEXT, Nullable)
    \item created\_at (TIMESTAMP, Default: CURRENT\_TIMESTAMP)
  \end{itemize}
  \item \textbf{Relationships:}
  \begin{itemize}
    \item One-to-one with user\_settings
    \item One-to-many with habits, tasks, busy\_schedules, progress, notifications
    \item One-to-many with assistant\_memories, calendar\_integrations, calendar\_events
    \item Many-to-many with achievements (via user\_achievements)
    \item Many-to-many with group\_challenges (via user\_group\_challenges)
    \item Self-referential relationship via friends
  \end{itemize}
\end{itemize}

\medskip
\textbf{User Settings Table}
\begin{itemize}
  \item \textbf{Table Name:} user\_settings
  \item \textbf{Fields:}
  \begin{itemize}
    \item id (SERIAL, Primary Key)
    \item user\_id (INT, FK $\rightarrow$ users.id, Unique, Not Null)
    \item timezone (VARCHAR(80), Default: UTC)
    \item daily\_reminder\_time (VARCHAR(10), Nullable)
    \item weekly\_summary\_day (VARCHAR(16), Default: Sunday)
    \item email\_notifications (BOOLEAN, Default: TRUE)
    \item push\_notifications (BOOLEAN, Default: FALSE)
    \item share\_activity (BOOLEAN, Default: TRUE)
    \item theme (VARCHAR(20), Default: light)
    \item ai\_tone (VARCHAR(30), Default: balanced)
    \item support\_style (VARCHAR(30), Default: celebrate)
    \item google\_calendar (BOOLEAN, Default: FALSE)
    \item apple\_calendar (BOOLEAN, Default: FALSE)
    \item fitness\_sync (BOOLEAN, Default: FALSE)
    \item created\_at (TIMESTAMP, Default: CURRENT\_TIMESTAMP)
  \end{itemize}
  \item \textbf{Relationships:}
  \begin{itemize}
    \item One-to-one with users
  \end{itemize}
\end{itemize}

\medskip
\textbf{Registration Verifications Table}
\begin{itemize}
  \item \textbf{Table Name:} registration\_verifications
  \item \textbf{Fields:}
  \begin{itemize}
    \item id (SERIAL, Primary Key)
    \item email (VARCHAR(150), Unique, Not Null)
    \item code\_hash (VARCHAR(200), Not Null)
    \item payload (JSONB, Not Null)
    \item expires\_at (TIMESTAMP, Not Null)
  \end{itemize}
\end{itemize}

\medskip
\textbf{Password Resets Table}
\begin{itemize}
  \item \textbf{Table Name:} password\_resets
  \item \textbf{Fields:}
  \begin{itemize}
    \item id (SERIAL, Primary Key)
    \item email (VARCHAR(150), Unique, Not Null)
    \item code\_hash (VARCHAR(200), Not Null)
    \item expires\_at (TIMESTAMP, Not Null)
  \end{itemize}
\end{itemize}

\medskip
\textbf{Habits Table}
\begin{itemize}
  \item \textbf{Table Name:} habits
  \item \textbf{Fields:}
  \begin{itemize}
    \item id (SERIAL, Primary Key)
    \item user\_id (INT, FK $\rightarrow$ users.id, Not Null)
    \item title (VARCHAR(100), Not Null)
    \item description (TEXT, Nullable)
    \item category (VARCHAR(50), Nullable)
    \item target\_reps (INT, Nullable)
    \item is\_daily\_goal (BOOLEAN, Default: FALSE)
    \item created\_at (TIMESTAMP, Default: CURRENT\_TIMESTAMP)
  \end{itemize}
  \item \textbf{Relationships:}
  \begin{itemize}
    \item Many-to-one with users
    \item One-to-many with schedules and progress
  \end{itemize}
\end{itemize}

\medskip
\textbf{Schedules Table}
\begin{itemize}
  \item \textbf{Table Name:} schedules
  \item \textbf{Fields:}
  \begin{itemize}
    \item id (SERIAL, Primary Key)
    \item habit\_id (INT, FK $\rightarrow$ habits.id, Not Null)
    \item user\_id (INT, FK $\rightarrow$ users.id, Not Null)
    \item day (DATE, Not Null)
    \item starttime (TIME, Not Null)
    \item endtime (TIME, Nullable)
    \item enddate (DATE, Nullable)
    \item repeat (VARCHAR(50), Default: daily)
    \item customdays (VARCHAR(100), Nullable)
    \item notes (TEXT, Nullable)
    \item created\_at (TIMESTAMP)
    \item updated\_at (TIMESTAMP)
  \end{itemize}
\end{itemize}

\medskip
\textbf{Tasks Table}
\begin{itemize}
  \item \textbf{Table Name:} tasks
  \item \textbf{Fields:}
  \begin{itemize}
    \item id (SERIAL, Primary Key)
    \item user\_id (INT, FK $\rightarrow$ users.id, Not Null)
    \item name (VARCHAR(255), Not Null)
    \item duration\_minutes (INT, Default: 60)
    \item min\_duration\_minutes (INT, Nullable)
    \item max\_duration\_minutes (INT, Nullable)
    \item split\_up (BOOLEAN, Default: FALSE)
    \item hours\_label (VARCHAR(120), Nullable)
    \item schedule\_after (TIMESTAMP, Nullable)
    \item due\_date (TIMESTAMP, Nullable)
    \item color (VARCHAR(20), Nullable)
    \item status (VARCHAR(20), Default: pending)
    \item created\_at (TIMESTAMP, Default: CURRENT\_TIMESTAMP)
  \end{itemize}
\end{itemize}

\medskip
\textbf{Busy Schedules Table}
\begin{itemize}
  \item \textbf{Table Name:} busy\_schedules
  \item \textbf{Fields:}
  \begin{itemize}
    \item id (SERIAL, Primary Key)
    \item user\_id (INT, FK $\rightarrow$ users.id, Not Null)
    \item title (VARCHAR(255), Not Null)
    \item day (DATE, Not Null)
    \item starttime (TIME, Not Null)
    \item endtime (TIME, Nullable)
    \item enddate (DATE, Nullable)
    \item repeat (VARCHAR(50), Default: daily)
    \item customdays (VARCHAR(100), Nullable)
    \item notes (TEXT, Nullable)
    \item created\_at (TIMESTAMP)
    \item updated\_at (TIMESTAMP)
  \end{itemize}
\end{itemize}

\medskip
\textbf{Progress Table}
\begin{itemize}
  \item \textbf{Table Name:} progress
  \item \textbf{Fields:}
  \begin{itemize}
    \item id (SERIAL, Primary Key)
    \item user\_id (INT, FK $\rightarrow$ users.id, Not Null)
    \item habit\_id (INT, FK $\rightarrow$ habits.id, Not Null)
    \item status (VARCHAR(50), Not Null)
    \item reflected\_reason (TEXT, Nullable)
    \item progress\_date (DATE, Default: CURRENT\_DATE)
    \item created\_at (TIMESTAMP)
  \end{itemize}
\end{itemize}

\medskip
\textbf{Achievements Table}
\begin{itemize}
  \item \textbf{Table Name:} achievements
  \item \textbf{Fields:}
  \begin{itemize}
    \item id (SERIAL, Primary Key)
    \item title (VARCHAR(100), Not Null)
    \item description (TEXT, Not Null)
    \item created\_at (TIMESTAMP)
  \end{itemize}
  \item \textbf{Relationships:}
  \begin{itemize}
    \item Many-to-many with users (via user\_achievements)
  \end{itemize}
\end{itemize}

\medskip
\textbf{User Achievements Table}
\begin{itemize}
  \item \textbf{Table Name:} user\_achievements
  \item \textbf{Fields:}
  \begin{itemize}
    \item id (SERIAL, Primary Key)
    \item user\_id (INT, FK $\rightarrow$ users.id, Not Null)
    \item achievement\_id (INT, FK $\rightarrow$ achievements.id, Not Null)
    \item achieved\_at (TIMESTAMP)
  \end{itemize}
  \item \textbf{Relationships:}
  \begin{itemize}
    \item Many-to-one with users
    \item Many-to-one with achievements
  \end{itemize}
\end{itemize}
\subsection{Data Security Measures}
The StepHabit platform implements multiple layers of security to protect user data,
ensure privacy, and maintain system integrity across authentication, authorization,
and data processing workflows.

Password Hashing.
User passwords are securely hashed using the \texttt{bcrypt} algorithm before storage.
This approach ensures that raw credentials are never persisted in the database and
protects user accounts in the event of a data breach.

JWT-Based Authentication.
Authentication is managed using JSON Web Tokens (JWT), which are issued upon successful
login and attached to subsequent API requests. Tokens are validated on protected
endpoints to ensure that only authenticated users can access sensitive resources.

Two-Step Verification.
StepHabit employs a code-based email verification mechanism during registration and
account recovery workflows. Verification codes are time-limited and must be validated
before account activation or password reset is completed.

Access Control and Authorization.
The system enforces strict ownership-based access control. Users can only view and
modify resources they own, such as habits, tasks, schedules, and progress logs. Public
features are limited to non-sensitive content, while all core functionality requires
authentication. This model ensures separation between authenticated users and
unauthenticated visitors.

Rate Limiting and Throttling.
To protect against brute-force attacks, verification workflows are rate-limited.
Each user is allowed a maximum of three verification code attempts within a one-minute
window. If this threshold is exceeded, the current verification code becomes invalid
and a new code must be issued.

HTTPS and CORS Policies.
All API communication is secured using HTTPS to protect data in transit. Cross-Origin
Resource Sharing (CORS) policies are configured to restrict access to trusted frontend
origins and prevent unauthorized cross-site requests.

Field-Level Validation and Sanitization.
All incoming API payloads undergo strict validation and sanitization at the controller
level. This prevents common injection attacks and ensures that only well-formed data
is processed by the application.

Privacy Controls and Visibility Boundaries.
StepHabit enforces strict visibility rules for user-generated content. Habits, tasks,
progress logs, and AI-generated insights are private by default and accessible only to
their respective owners unless explicitly shared through controlled social features.

Transactional Integrity.
The application ensures atomic data operations by grouping dependent database actions
within transactional boundaries. This guarantees that multi-step operations—such as
creating habits, schedules, progress records, or social relationships—are either fully
completed or fully rolled back in the event of an error, preserving database
consistency.


\newpage

% ===========================
% Chapter 4
% ===========================
\section{Backend Endpoints}
\subsection{Habit Endpoints}
\begin{longtable}{|p{3cm}|p{4cm}|p{6cm}|p{2.5cm}|}
\hline
\textbf{Method} & \textbf{Path} & \textbf{Description} & \textbf{Auth Required} \\ \hline
GET & /habits & Retrieve all habits for the authenticated user & Yes \\ \hline
GET & /habits/:id & Retrieve a habit by its identifier & Yes \\ \hline
POST & /habits & Create a new habit & Yes \\ \hline
PUT & /habits/:id & Update an existing habit & Yes \\ \hline
DELETE & /habits/:id & Delete a habit and related data & Yes \\ \hline
POST & /habits/ai/plan & Generate AI-based habit plan & Yes \\ \hline
POST & /habits/ai/rewrite & Rewrite habit description using AI & Yes \\ \hline
\end{longtable}
\subsection{Task Endpoints}
\begin{longtable}{|p{3cm}|p{4cm}|p{6cm}|p{2.5cm}|}
\hline
\textbf{Method} & \textbf{Path} & \textbf{Description} & \textbf{Auth Required} \\ \hline
GET & /tasks & Retrieve all tasks for the user & Yes \\ \hline
POST & /tasks & Create a new task & Yes \\ \hline
PUT & /tasks/:id & Update task details & Yes \\ \hline
PATCH & /tasks/:id/status & Update task status & Yes \\ \hline
DELETE & /tasks/:id & Delete a task & Yes \\ \hline
\end{longtable}
\subsection{Progress Tracking Endpoints}
\begin{longtable}{|p{3cm}|p{4cm}|p{6cm}|p{2.5cm}|}
\hline
\textbf{Method} & \textbf{Path} & \textbf{Description} & \textbf{Auth Required} \\ \hline
GET & /progress & Retrieve habit progress logs & Yes \\ \hline
POST & /progress & Log habit completion & Yes \\ \hline
DELETE & /progress/:id & Delete a progress record & Yes \\ \hline
\end{longtable}
\subsection{User & Authentication Endpoints}
\begin{longtable}{|p{3cm}|p{4cm}|p{6cm}|p{2.5cm}|}
\hline
\textbf{Method} & \textbf{Path} & \textbf{Description} & \textbf{Auth Required} \\ \hline
POST & /users/register & Register a new user & No \\ \hline
PATCH & /users/register/verify & Complete registration via code & No \\ \hline
POST & /users/login & Login with email and password & No \\ \hline
PATCH & /users/login/verify & Login using verification code & No \\ \hline
GET & /users/me & Get authenticated user profile & Yes \\ \hline
PUT & /users/:id & Update user profile & Yes \\ \hline
DELETE & /users & Delete user account & Yes \\ \hline
POST & /users/logout & Logout authenticated user & Yes \\ \hline
POST & /users/forgot-password & Request password reset & No \\ \hline
PATCH & /users/reset-password & Reset password via code & Yes \\ \hline
\end{longtable}
\subsection{Notification Endpoints}
\begin{longtable}{|p{3cm}|p{4cm}|p{6cm}|p{2.5cm}|}
\hline
\textbf{Method} & \textbf{Path} & \textbf{Description} & \textbf{Auth Required} \\ \hline
GET & /notifications & Retrieve user notifications & Yes \\ \hline
POST & /notifications & Create a notification & Yes \\ \hline
PATCH & /notifications/mark-all-read & Mark all notifications as read & Yes \\ \hline
PATCH & /notifications/:id/read & Mark a notification as read & Yes \\ \hline
\end{longtable}

\subsection{Messaging & Social Endpoints}
\begin{longtable}{|p{3cm}|p{4cm}|p{6cm}|p{2.5cm}|}
\hline
\textbf{Method} & \textbf{Path} & \textbf{Description} & \textbf{Auth Required} \\ \hline
GET & /messages & Retrieve chat messages & Yes \\ \hline
POST & /messages & Send a direct message & Yes \\ \hline
POST & /group-challenges & Create a group challenge & Yes \\ \hline
GET & /group-challenges & List group challenges & Yes \\ \hline
POST & /group-challenges/:id/join & Join a challenge & Yes \\ \hline
\end{longtable}

\subsection{WebSocket Endpoint}
\begin{longtable}{|p{5cm}|p{9cm}|}
\hline
\textbf{WebSocket URL} & ws://<HOST>:<PORT>/notification-message \\ \hline
\textbf{Namespace} & /notification-message \\ \hline
\textbf{Transport} & WebSocket \\ \hline
\textbf{Authentication} & JWT (Authorization Header) \\ \hline
\end{longtable}

\subsection{API Tools and Documentation}

To support development, testing, and long-term maintainability, the StepHabit platform
provides comprehensive API documentation and testing tools. These tools allow developers
and reviewers to explore endpoint behavior, request formats, and response structures in
a transparent and interactive manner.

\begin{longtable}{|p{4cm}|p{8cm}|p{3cm}|}
\hline
\textbf{Tool} & \textbf{Description} & \textbf{Access URL} \\ \hline
Swagger UI &
Interactive API documentation that exposes all available endpoints, including
request parameters, authentication requirements, and response schemas. It enables
real-time testing of API calls directly from the browser. &
\texttt{/api\#} \\ \hline
Postman Collection &
A curated Postman collection used for manual API testing and debugging during
development. It includes preconfigured requests for authentication, habits, tasks,
notifications, AI coaching, and social features. &
\texttt{Postman Workspace} \\ \hline
\end{longtable}


\newpage

% ===========================
% Chapter 5
% ===========================
\section{Application and functionality}

\subsection{User Registration Process}

The user registration workflow in StepHabit is designed to ensure account integrity,
data validity, and protection against unauthorized access. The process is divided
into three structured stages, each incorporating validation rules and security
mechanisms to maintain platform reliability.
\begin{enumerate}[label=\textbf{Step \arabic*:}, leftmargin=*]
  \item \textbf{Account Information Entry}\\
  The registration process begins with the user providing the required account
  fields: username, email address, and password. The system enforces validation to
  keep usernames unique, reject invalid or duplicate email addresses, and require a
  minimum password length of eight characters. If any rule fails, descriptive errors
  block progression until the input is corrected.

  \begin{figure}[H]
  \centering
  \includegraphics[width=0.75\textwidth]{images/registration_step1_form.png}
  \caption{StepHabit registration form for entering account information}
  \end{figure}

  \item \textbf{Goal and Focus Area Selection}\\
  After credentials are validated, onboarding captures the user’s motivation and
  preferred focus areas to personalize recommendations, dashboards, and AI coaching.
  Users select a primary motivation (e.g., building consistency, boosting energy,
  improving focus and clarity, achieving balance and wellbeing) and an initial focus
  area (e.g., mindfulness, fitness, productivity, self-care). These selections are
  stored for downstream scheduling and coaching.

  \begin{figure}[H]
  \centering
  \begin{minipage}[t]{0.47\textwidth}
    \centering
    \includegraphics[width=0.82\textwidth]{images/onboarding_step2_goal.png}
    \caption{Selection of primary motivation and personal goal}
  \end{minipage}
  \hfill
  \begin{minipage}[t]{0.47\textwidth}
    \centering
    \includegraphics[width=0.82\textwidth]{images/onboarding_step2_focus.png}
    \caption{Selection of initial focus area}
  \end{minipage}
  \end{figure}

  \item \textbf{Commitment Level and Support Preferences}\\
  The third onboarding stage captures time commitment, experience level, and support
  style so habit intensity, reminder frequency, and AI tone match the user’s
  expectations. Users choose daily time dedication (e.g., five, fifteen, thirty
  minutes, or flexible), indicate their habit-building stage from beginner to
  advanced, and pick a support style such as gentle nudges, focused reminders, deep
  insights, or celebratory feedback. An optional motivational statement is persisted
  for future AI reference.

  \item \textbf{Verification Code Validation}\\
  After account details are submitted, StepHabit sends a time-limited verification
  code to confirm email ownership. The validation workflow is rate-limited to protect
  against brute-force attacks and unauthorized account creation attempts.

  \begin{figure}[H]
  \centering
  \includegraphics[width=0.75\textwidth]{images/registration_step3_profile.png}
  \caption{Optional profile initialization for personalized habit coaching}
  \end{figure}

  \item \textbf{Profile Initialization}\\
  Once verification succeeds, the account is activated and the user can optionally
  set profile preferences that tailor the StepHabit experience and AI coaching. Users
  may provide a primary goal and focus area, daily commitment level, and preferred
  support or coaching style.
\end{enumerate}

\subsubsection*{Registration Completion}

Upon completing all required steps, the user is redirected to the main dashboard
and granted full access to the platform’s core functionality, including habit
creation, task planning, scheduling, and AI-assisted coaching.
\begin{figure}[H]
\centering
\includegraphics[width=0.75\textwidth]{images/dashboard.png}
\caption{Successful registration and initial access to the StepHabit dashboard}
\end{figure}

\subsubsection*{Dashboard (Personal Hub)}

The Dashboard is the primary landing screen after authentication. It provides a
high-level snapshot of the user’s daily progress, shortcuts to core actions, and
AI-assisted guidance. The page is designed to minimize friction by letting users
create habits, schedule time blocks, and log progress directly from the main hub.

Key elements on this screen include: (i) the \textit{Personal Hub} welcome panel
with action buttons (\textit{Add Habit}, \textit{Add Schedule}, \textit{Log Progress},
\textit{AI Summary}), (ii) a \textit{Today’s completion} progress indicator, (iii)
summary KPI cards (check-ins, completion rate, active habits, best streak), and
(iv) contextual widgets such as \textit{Next up today}, \textit{Momentum snapshot},
and a \textit{Daily AI Tip}. Together, these components guide the user toward
consistent daily execution while maintaining visibility into progress.
\begin{figure}[H]
\centering
\includegraphics[width=0.82\textwidth]{images/dashboard_overview.png}
\caption{StepHabit Dashboard (Personal Hub) with progress indicators, quick actions, and AI guidance}
\end{figure}

\subsection{Planner Module}
The Planner is the central scheduling component of StepHabit. It enables users to
design structured daily and weekly routines by combining habit-related time blocks,
busy periods, and externally imported calendar events into a single, unified view.
This section plays a critical role in bridging long-term habit goals with concrete
time allocation.
At the top of the Planner, the system presents a high-level status overview,
including the current focus date, overall planning health, and calendar synchronization
status. Users can create new time blocks, synchronize external calendars, or navigate
directly to their saved schedules.
\begin{figure}[H]
\centering
\includegraphics[width=0.82\textwidth]{images/planner_overview.png}
\caption{Unified Planner overview with planning health indicators and schedule controls}
\end{figure}
The core of the Planner is the routine view, which combines a monthly calendar
overview with a detailed daily timeline. Saved time blocks and imported calendar
events are visually distinguished, allowing users to quickly identify free windows
and scheduling conflicts. Selecting a specific day reveals all associated routines
and busy periods for that date.
\begin{figure}[H]
\centering
\includegraphics[width=0.82\textwidth]{images/planner_calendar_day.png}
\caption{Calendar overview and daily schedule with saved time blocks}
\end{figure}
Users can add new schedule entries through the \textit{Add Schedule} interface.
Schedules can either be linked to a specific habit or marked as busy events.
Linking a schedule to a habit allows the AI assistant to understand when routines
are expected to occur, enabling more accurate habit suggestions and progress
tracking. Busy events, on the other hand, inform the system about unavailable
time windows and are not treated as completion-based activities.
\begin{figure}[H]
\centering
\includegraphics[width=0.82\textwidth]{images/planner_add_schedule.png}
\caption{Creating and managing schedule entries and busy time blocks}
\end{figure}
The Planner also supports calendar synchronization through external providers
such as Google Calendar. Users can upload calendar data using \texttt{.ics} files
or connect accounts directly. Imported events are stored separately from habits
and schedules but are displayed alongside them to prevent conflicts and enable
smarter time allocation.
\begin{figure}[H]
\centering
\includegraphics[width=0.82\textwidth]{images/planner_calendar_sync.png}
\caption{External calendar integration and upcoming event synchronization}
\end{figure}
All planner data, including schedules, busy blocks, and calendar events, is
persisted in the database and associated with the authenticated user. This allows
the AI assistant, notification system, and progress analytics modules to operate
on a shared, consistent view of the user’s time and routines.
\subsection{Tasks Module}

The Tasks module allows users to capture, organize, and prepare actionable tasks
before they are scheduled into the planner. This separation enables users to think
clearly about task scope, duration, and deadlines without immediately committing
to a specific time block.

\subsubsection*{Task Board Overview}

The Task Board provides a centralized list of all user-created tasks. From this
interface, users can view their tasks, create new ones, and track progress at a
high level.
\begin{figure}[H]
\centering
\includegraphics[width=0.75\textwidth]{images/tasks_empty_board.png}
\caption{Initial task board view when no tasks are present}
\end{figure}
When no tasks exist, the system displays a clear call-to-action prompting the user
to create their first task.
\subsubsection*{Creating a New Task}
Users can add a new task by clicking the \textit{New Task} button, which opens a
modal form. This form collects structured task metadata required for intelligent
scheduling and planning.
\begin{figure}[H]
\centering
\includegraphics[width=0.75\textwidth]{images/add_task_modal.png}
\caption{Task creation modal with duration and scheduling constraints}
\end{figure}
The task creation form includes the following configurable fields:
\begin{itemize}
  \item Task name
  \item Estimated duration (in minutes)
  \item Minimum and maximum allowed duration
  \item Optional task splitting for flexible scheduling
  \item Preferred working hours
  \item Optional scheduling constraint (schedule after)
  \item Due date
  \item Visual task color for identification
\end{itemize}
Once submitted, the task is persisted and becomes visible in the task board.
\subsubsection*{Task List and Feedback}
After a task is created, the system provides immediate visual feedback confirming
successful creation. Tasks are displayed in a reorderable list, allowing users to
prioritize items manually.
\begin{figure}[H]
\centering
\includegraphics[width=0.75\textwidth]{images/task_created_list.png}
\caption{Task board after successful task creation}
\end{figure}
Tasks can be reordered via drag-and-drop, marked as completed or missed, or opened
for further editing.
\subsubsection*{Task Details and Checklists}
Clicking on a task opens a detailed editor where users can enrich the task with
descriptions and actionable checklists. This enables breaking down complex tasks
into smaller, trackable steps.
\begin{figure}[H]
\centering
\includegraphics[width=0.75\textwidth]{images/task_details_modal.png}
\caption{Task detail view with checklist and progress tracking}
\end{figure}
Each checklist item can be individually marked as complete, with the interface
displaying a real-time progress indicator to reflect overall task completion.
This design supports both lightweight task tracking and deeper task decomposition,
depending on user needs.
\subsection{Habits Module}
The Habits module is the core component of the StepHabit platform. It allows users to
create, manage, track, and analyze daily habits while receiving structured insights and
AI-supported guidance. The interface is designed to remain calm, minimal, and data-driven,
encouraging consistency and long-term behavior change.
\subsubsection*{Habits Overview}
The Habits overview page serves as the central hub for habit-related activity. It provides
a summary of the user's current progress, including weekly win rate, current streak, and
the total number of active habits. From this page, users can navigate to habit creation,
the habit library, progress analytics, AI coaching, insights, and historical logs.
\begin{figure}[H]
\centering
\includegraphics[width=0.82\textwidth]{images/habits_overview.png}
\caption{Habits overview dashboard showing summary statistics and navigation tabs}
\end{figure}
\subsubsection*{Habit Creation}
Users can create a new habit using the structured habit creation form. This form collects
essential information such as the habit title, description, category, and target
repetitions. An AI-powered habit rewriter is also available, allowing users to input rough
ideas and receive refined habit definitions.
\begin{figure}[H]
\centering
\includegraphics[width=0.82\textwidth]{images/add_habit_form.png}
\caption{Habit creation interface with AI habit rewriting and live preview}
\end{figure}
\subsubsection*{Habit Library}
The Habit Library provides a curated collection of pre-designed habits that users can add
to their routine with a single click. Habits are categorized by focus area, difficulty,
and time of day. Popular and trending habits are highlighted to assist users in discovery.
\begin{figure}[H]
\centering
\includegraphics[width=0.82\textwidth]{images/habit_library.png}
\caption{Habit library with filters, curated habits, and trending recommendations}
\end{figure}
\subsubsection*{Progress Tracking}
The Progress section visualizes habit performance over time. Users can analyze weekly,
monthly, or yearly completion rates, monitor streaks, and identify their most productive
days. This data-driven view reinforces consistency and helps users adjust their routines.
\begin{figure}[H]
\centering
\includegraphics[width=0.82\textwidth]{images/habit_progress.png}
\caption{Habit progress tracker with completion rates and streak analysis}
\end{figure}
\subsubsection*{Insights and Analytics}
The Insights page transforms logged habit data into meaningful feedback. It highlights
high-performing habits, identifies habits needing attention, and provides recommendations
for optimization. Insights are refreshed dynamically as users log progress.
\begin{figure}[H]
\centering
\includegraphics[width=0.82\textwidth]{images/habit_insights.png}
\caption{Insights dashboard showing habit performance analysis and recommendations}
\end{figure}
\subsubsection*{Habit History}
The History section maintains a chronological record of the user's habit check-ins,
including optional reflections. Users can filter by habit and date range or export data
for external analysis.
\begin{figure}[H]
\centering
\includegraphics[width=0.82\textwidth]{images/habit_history.png}
\caption{Habit history view with filters and export functionality}
\end{figure}
\subsection{HabitCoach(AI Assistant) Module}
\subsection{HabitCoach: AI-Powered Personal Assistant}
HabitCoach is an intelligent conversational assistant integrated deeply into the StepHabit
platform. Unlike a traditional chatbot, HabitCoach has full contextual awareness of the
user’s profile, habits, schedules, tasks, calendar events, and historical activity. This
allows it to act as a personalized habit and planning assistant rather than a generic
question–answer system.
HabitCoach continuously follows the user’s conversation context and can safely access all
relevant database tables, including habits, schedules, tasks, progress logs, notifications,
and calendar integrations. This enables it to perform direct actions on behalf of the user
instead of only providing suggestions.
\subsubsection*{Smart Scheduling and Reminders}
HabitCoach can create, modify, and manage schedules through natural language commands.
Users can request recurring events, time blocks, or habit-linked activities without manual
form interaction. For example, a user can ask HabitCoach to schedule a daily class or
recurring routine across a specific date range.
In addition to scheduling, HabitCoach automatically configures reminders based on context.
Reminder behavior is adaptive and priority-aware:
\begin{itemize}
  \item For structured events such as classes or meetings, reminders are scheduled earlier
  (e.g., one hour before the session).
  \item For habits and short routines, reminders may be delivered closer to execution
  time (e.g., five minutes before).
  \item Reminder priority is adjusted based on event type, user preferences, and habit
  importance.
\end{itemize}
\subsubsection*{Personalization and Learning}
HabitCoach continuously learns from user behavior, preferences, and interaction patterns.
As users log habits, complete tasks, miss routines, or reschedule activities, the assistant
refines its understanding of the user’s lifestyle and energy patterns.
Over time, this learning process allows HabitCoach to:
\begin{itemize}
  \item Suggest optimal scheduling windows based on past consistency.
  \item Recommend habit adjustments when performance drops.
  \item Personalize reminder timing and tone.
  \item Provide increasingly relevant guidance and next-step suggestions.
\end{itemize}
\subsubsection*{Conversational Control}
HabitCoach supports free-form natural language interaction. Users can communicate with the
assistant as they would with a human coach, asking questions, requesting changes, or
seeking advice. The assistant responds in a supportive, encouraging tone while maintaining
clear confirmation of actions performed.
This conversational layer reduces friction and lowers the cognitive barrier for planning,
making the system accessible even for users who prefer not to interact with complex
interfaces.
\begin{figure}[H]
\centering
\includegraphics[width=0.82\textwidth]{images/habitcoach_chat.png}
\caption{HabitCoach scheduling a recurring class and configuring a smart reminder through conversation}
\end{figure}
\subsection{Communiy Module}
The Community module of StepHabit provides a controlled social layer that enhances
accountability, motivation, and long-term habit adherence. Rather than functioning as a
traditional social network, this module is designed around purposeful interaction,
progress sharing, and collaborative challenges, while maintaining strong privacy controls.
\subsubsection*{Friends Management}
Users can discover, add, and manage friends through the Community interface. Friend
connections are established via searchable identifiers such as name or email and require
explicit acceptance, ensuring consent-based social interaction.
Once connected, users can selectively share habit activity with friends. Sharing
preferences are configurable per friend, allowing users to control whether their habits
and progress are visible. This enables accountability partnerships without compromising
personal boundaries.
\begin{figure}[H]
\centering
\includegraphics[width=0.82\textwidth]{images/community_friends.png}
\caption{Community friends view with pending requests, search, and sharing controls}
\end{figure}
\subsubsection*{Private Messaging}
The messaging system allows users to communicate directly with their connected friends.
Conversations are organized into clean, threaded dialogs that support text messages and
optional attachments such as images or location data.
This feature is intended to support encouragement, progress updates, and habit-related
discussions rather than high-volume social interaction. Messages are private and accessible
only to authenticated participants in the conversation.
\begin{figure}[H]
\centering
\includegraphics[width=0.82\textwidth]{images/community_messages.png}
\caption{Direct messaging interface between connected users}
\end{figure}
\subsubsection*{Group Challenges}
The Challenges feature allows users to create or join group-based goals designed to promote
collective accountability. A challenge includes a title, date range, optional approval
requirements, and a host who manages participation.
Participants can join challenges, track progress together, and communicate through a
dedicated challenge chat. This structure encourages social motivation while keeping the
focus on shared objectives rather than competition.
\begin{figure}[H]
\centering
\includegraphics[width=0.82\textwidth]{images/community_challenges.png}
\caption{Group challenges overview with joinable challenges and challenge creation}
\end{figure}
\subsubsection*{Challenge Chat}
Each group challenge includes an embedded chat interface where members can share updates,
encouragement, or progress milestones. The challenge chat is isolated to the group and
distinct from direct messages, ensuring contextual and goal-oriented communication.
This communication channel strengthens group cohesion and reinforces accountability
through shared progress visibility.
\subsection{Profile Module}
The Profile module serves as the centralized control center for user identity,
preferences, personalization, and system integrations. It allows users to manage
their account information, configure how the platform behaves, and fine-tune how
AI-driven features such as HabitCoach interact with them.
This module is designed to balance personalization, transparency, and user control,
ensuring that recommendations and reminders remain aligned with individual goals
and preferences.
\subsubsection*{Account Information}
Users can manage their core account details, including display name, email address,
and optional demographic information. Profile completion progress is visualized to
encourage users to provide sufficient data for better personalization while keeping
all non-essential fields optional.
Users can upload a profile avatar and securely reset their password from this section.
Sensitive operations such as password updates are protected by authentication and
server-side validation.
\begin{figure}[H]
\centering
\includegraphics[width=0.82\textwidth]{images/profile_account.png}
\caption{Profile account management with completion progress and avatar upload}
\end{figure}
\subsubsection*{User Preferences}
The Preferences tab allows users to shape how StepHabit feels and responds. Users can
select visual themes (e.g., light or dark mode), configure the tone of AI interactions,
and choose a preferred support style such as encouragement or celebration of wins.
These settings directly influence how HabitCoach communicates, ensuring that feedback
and suggestions match the user’s motivational style.
\begin{figure}[H]
\centering
\includegraphics[width=0.82\textwidth]{images/profile_preferences.png}
\caption{User preference controls for theme, AI tone, and support style}
\end{figure}
\subsubsection*{Notification Settings}
Users can configure how and when they receive notifications. The system supports
email and push-based reminders, with smart spacing to avoid notification fatigue.
Notifications are context-aware and adapt based on habit type, importance, and user
behavior. For example, critical events may trigger earlier reminders, while routine
habits generate lighter prompts closer to execution time.
\begin{figure}[H]
\centering
\includegraphics[width=0.82\textwidth]{images/profile_notifications.png}
\caption{Notification preferences with smart reminder controls}
\end{figure}
\subsubsection*{Connected Applications}
The Connected Apps section enables integration with external services such as calendar
providers and fitness platforms. Calendar synchronization allows StepHabit to plan
habits around real-world commitments, avoiding scheduling conflicts.
Some integrations are active, while others are marked as future extensions. This
design allows the system to remain extensible without disrupting the current user
experience.
\begin{figure}[H]
\centering
\includegraphics[width=0.82\textwidth]{images/profile_connected_apps.png}
\caption{Connected applications for calendar and fitness synchronization}
\end{figure}
\subsubsection*{Help and Support}
The Help \& Support section provides users with access to documentation, FAQs, and
direct communication channels with the development team. This ensures that users
can resolve issues, learn system features, and provide feedback when needed.
Support content is integrated directly into the platform to minimize disruption and
maintain user trust.
\begin{figure}[H]
\centering
\includegraphics[width=0.82\textwidth]{images/profile_help_support.png}
\caption{Help center and contact options within the profile module}
\end{figure}
\subsubsection*{Role of Profile Data in AI Personalization}
Profile settings directly influence the behavior of AI-driven components such as
HabitCoach and smart reminders. By combining explicit preferences with observed
behavior, the system continuously refines scheduling suggestions, reminder timing,
and communication style.
This feedback loop ensures that StepHabit adapts to the user over time while remaining
transparent and user-controlled.
\subsection{Notification Center and Smart Reminder Module}
The Notification Center acts as the central hub for all system-generated and user-created
alerts within StepHabit. It consolidates reminders, progress nudges, schedule alerts, and
AI-driven notifications into a single, organized interface, allowing users to stay informed
without being overwhelmed.
Notifications are categorized, prioritized, and timestamped, giving users full visibility
into what requires attention, what is upcoming, and what has already been acknowledged.
\subsubsection*{Notification Overview}
At the top level, the Notification Center provides a real-time summary of notification
activity, including total notifications, unread items, upcoming scheduled reminders, and
the most recently opened alert. Users can filter notifications by status or category to
quickly locate relevant information.
This design ensures clarity and reduces cognitive load, especially for users managing
multiple habits, schedules, and reminders.
\begin{figure}[H]
\centering
\includegraphics[width=0.82\textwidth]{images/notification_center_overview.png}
\caption{Notification Center overview with status indicators and filtering options}
\end{figure}
\newpage

% ===========================
% Chapter 6
% ===========================
\section{Challenges and Lessons Learned}
Developing StepHabit as a full-stack, AI-assisted habit management platform introduced
several technical and architectural challenges. Addressing these challenges required
careful system design, iteration, and trade-off analysis, ultimately contributing to a
deeper understanding of real-world software engineering practices.
\subsection{Real-Time Communication and Event Handling}
One of the main challenges was implementing real-time features that support instant
feedback and synchronization across the application. StepHabit uses real-time
communication mechanisms to support features such as notifications, HabitCoach
responses, and dynamic UI updates.
Managing user sessions in real time required handling edge cases such as unexpected
disconnections, reconnections, and multiple active sessions. Ensuring consistent event
delivery and avoiding duplicate or missed updates was particularly challenging when users
navigated across different sections of the application.
\textbf{Lesson learned:} This challenge strengthened my understanding of real-time
client--server synchronization, event lifecycle management, and the importance of
designing efficient and well-structured payloads for scalable communication.
\subsection{Frontend and Backend Integration}
Integrating the frontend and backend layers required maintaining strict consistency across
data models, validation rules, and API contracts. Several issues emerged when frontend
expectations did not fully align with backend responses, particularly during authentication,
form validation, habit creation flows, and error handling.
Ensuring predictable behavior required refining API response formats and improving
validation feedback so that errors could be clearly communicated to the user interface.
\textbf{Lesson learned:} Clearly documented APIs and a shared data contract between the
frontend and backend are essential for reliable integration. Tools such as Swagger were
critical in aligning expectations and accelerating development.
\subsection{AI Integration and User Experience}
Integrating the HabitCoach AI presented both technical and design challenges. The AI
system needed access to user context, habits, schedules, and preferences while maintaining
performance, safety, and responsiveness. Handling AI latency, fallback behavior, and usage
limits required careful coordination between the backend logic and frontend experience.
Additionally, AI-generated suggestions had to remain supportive, relevant, and trustworthy,
which required balancing automation with user control and transparency.
\textbf{Lesson learned:} AI integration extends beyond model outputs. It requires careful
UX design, safety considerations, rate limiting, and graceful degradation to maintain user
trust and system reliability.
\subsection{Scheduling Logic and Smart Notifications}
Designing the scheduling and reminder system was another significant challenge. StepHabit
supports both habit-based and event-based reminders, each requiring different timing,
priority levels, and delivery logic. For example, scheduled classes may trigger reminders
earlier, while habits require minimal but timely nudges.
Preventing notification fatigue while still providing effective guidance required
implementing priority-based logic and respecting user-defined notification preferences.
\textbf{Lesson learned:} Effective reminder systems must balance automation with
personalization. Smart defaults combined with user control significantly improve
engagement and usability.
\subsection{Overall Reflection}
Developing StepHabit highlighted the importance of system thinking, clear contracts between
components, and iterative refinement. Each challenge contributed to a deeper understanding
of scalable architecture, AI-assisted systems, and user-centered software design.
These experiences collectively reinforced the value of building robust, maintainable
systems that balance technical complexity with real-world usability.
\newpage

% ===========================
% Chapter 7
% ===========================
\section{Future Improvements}

While StepHabit already provides a comprehensive habit, scheduling, and AI-assisted
coaching experience, several enhancements can further improve scalability, usability,
and long-term impact.

\subsection{Mobile Application Development}

A natural next step for StepHabit is the development of native mobile applications for
iOS and Android platforms. Mobile access would enable real-time habit logging, instant
push notifications, and tighter integration with device features such as calendars,
health data, and location-based reminders.

\textbf{Planned improvement:} Develop a cross-platform mobile application (e.g., using
Flutter or React Native) that shares the same backend and authentication system to
ensure consistency across web and mobile experiences.

\subsection{Production Deployment and Cloud Infrastructure}

Currently, StepHabit is designed and tested in a development environment. Deploying the
application to a production-grade cloud infrastructure would significantly improve
reliability, availability, and performance for real users.

\textbf{Planned improvement:} Host the backend on a cloud platform with scalable
resources, configure environment-based deployments, and introduce monitoring and
logging to track system health and usage patterns.

\subsection{Advanced Personalization and Recommendation Engine}

Although StepHabit already provides AI-driven guidance through HabitCoach, future
iterations could introduce a dedicated recommendation engine that learns from user
behavior over time. This system could suggest habits, optimal scheduling windows, and
adjust reminder timing based on historical success rates and engagement patterns.

\textbf{Planned improvement:} Implement a behavior-aware recommendation system that
analyzes habit completion history, streak consistency, and time-of-day performance to
deliver increasingly personalized suggestions.

\subsection{Enhanced Notification Intelligence}

While the current notification system supports smart reminders, future versions could
introduce adaptive notification strategies that respond dynamically to user behavior.
For example, the system could reduce reminder frequency when habits are consistently
completed or increase support during periods of decline.

\textbf{Planned improvement:} Introduce adaptive reminder logic that balances motivation
with notification fatigue by learning optimal reminder timing and priority per user.

\subsection{Expanded AI Capabilities}

HabitCoach currently assists users with scheduling, reminders, and habit-related
guidance. Future enhancements could enable deeper conversational memory, long-term
behavior analysis, and proactive coaching strategies.

\textbf{Planned improvement:} Extend HabitCoach with long-term context awareness,
goal-based planning, and voice interaction support to create a more human-like and
supportive coaching experience.

\subsection{Scalability and Performance Optimization}

As the user base grows, optimizing performance becomes increasingly important. Certain
frequently accessed data—such as habit summaries, schedules, and insight metrics—can
be optimized to reduce response time and backend load.

\textbf{Planned improvement:} Introduce caching strategies and query optimization to
ensure the system remains responsive under increased traffic.

\subsection{Overall Vision}

These future improvements aim to evolve StepHabit from a feature-rich academic project
into a scalable, production-ready platform. By expanding to mobile, strengthening AI
personalization, and improving system performance, StepHabit has the potential to become
a long-term personal growth companion for users across different platforms.

\newpage

% ===========================
% Chapter 8
% ===========================
\section{Conclusion}

StepHabit represents a comprehensive and thoughtfully engineered solution for habit
formation, productivity planning, and long-term personal growth. By combining habit
tracking, task management, intelligent scheduling, and AI-powered coaching, the platform
supports users in building sustainable routines while maintaining clarity, balance, and
motivation in their daily lives. Features such as smart reminders, progress visualization,
streak tracking, and adaptive insights encourage consistency, self-awareness, and
accountability over time.

The system is built on a modular and scalable architecture that clearly separates concerns
across its core components. The frontend interface provides a clean, calm, and intuitive
user experience that enables users to plan schedules, manage tasks, track habits, and
interact with the AI HabitCoach seamlessly. The backend layer handles authentication,
business logic, data processing, and real-time operations securely and efficiently, ensuring
reliable communication between system components.

A central aspect of StepHabit is the integration of AI-driven assistance through
\textit{HabitCoach}. The AI assistant is capable of accessing user context, schedules,
habits, tasks, and preferences to suggest routines, create schedules, set smart reminders,
and provide personalized guidance. Unlike static reminder systems, HabitCoach adapts
notification timing and priority based on context—for example, reminding users earlier for
high-priority events such as classes, while using shorter reminders for habits. Over time,
the assistant learns user behavior and preferences, enabling a more supportive and
human-like coaching experience.

The platform also includes community-oriented features that enhance motivation through
social accountability. Users can connect with friends, exchange messages, participate in
group challenges, and share progress selectively. These features encourage collaboration,
peer support, and healthy competition, reinforcing habit consistency beyond individual
use.

StepHabit leverages a modern technology stack designed for scalability and future growth.
Secure authentication, real-time updates, structured APIs, and well-defined data models
ensure system reliability and maintainability. The development workflow is supported by
modern tooling and best practices, enabling efficient collaboration, debugging, and
iteration throughout the project lifecycle.

Looking ahead, StepHabit is well-positioned for further expansion. Planned improvements
include deploying the system to a production-grade server environment, developing native
mobile applications, enhancing AI personalization, and deepening integration with external
services such as calendars and fitness platforms. These enhancements aim to transform
StepHabit from a feature-rich academic project into a scalable, real-world productivity and
habit-building platform.

In conclusion, StepHabit demonstrates how modern web technologies, intelligent system
design, and AI-driven personalization can be combined to support meaningful behavior
change. By focusing on clarity, adaptability, and user-centered design, the platform
illustrates the potential of digital tools to help users build better habits, stay consistent,
and grow intentionally over time.
\newpage

\section*{References}

\begin{enumerate}
  \item Lally, P., van Jaarsveld, C. H. M., Potts, H. W. W., \& Wardle, J. (2010). 
  How are habits formed: Modelling habit formation in the real world. 
  \textit{European Journal of Social Psychology, 40}(6), 998--1009. 
  https://doi.org/10.1002/ejsp.674

  \item Stawarz, K., Cox, A. L., \& Blandford, A. (2015). 
  Beyond self-tracking and reminders: Designing smartphone apps that support habit formation. 
  \textit{Proceedings of the 33rd Annual ACM Conference on Human Factors in Computing Systems}, 2653--2662. 
  https://doi.org/10.1145/2702123.2702230

  \item Streaks. (n.d.). 
  \textit{Streaks: Habit tracking app}. 
  https://streaksapp.com

  \item Habitica. (n.d.). 
  \textit{Habitica: Gamified habit-building and productivity app}. 
  https://habitica.com

  \item Habitify. (n.d.). 
  \textit{Habitify: Habit tracker and routine planner}. 
  https://habitify.me

  \item Soron, A. (n.d.). 
  \textit{Loop habit tracker} [Open-source software]. GitHub. 
  https://github.com/iSoron/uhabits

  \item Zapier. (2025). 
  \textit{The best habit tracker apps}. 
  https://zapier.com/blog/best-habit-tracker-app/

  \item Node.js Foundation. (n.d.). 
  \textit{Node.js documentation}. 
  https://nodejs.org/en/docs

  \item IBM. (n.d.). 
  \textit{What is Claude AI and why it matters}. IBM Think. 
  https://www.ibm.com/think/topics/claude-ai

  \item PostgreSQL Global Development Group. (n.d.). 
  \textit{Why PostgreSQL? Advantages and features}. PostgreSQL Documentation. 
  https://www.postgresql.org/about/

  \item CoreUI. (n.d.). 
  \textit{CoreUI React admin template and UI components}. 
  https://coreui.io/react
\end{enumerate}

\end{document}
