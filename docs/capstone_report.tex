\documentclass[12pt]{article}
\usepackage[margin=1in]{geometry}
\usepackage{hyperref}
\usepackage{graphicx}
\usepackage{array}
\usepackage{longtable}
\usepackage{enumitem}
\usepackage{listings}
\usepackage{xcolor}
\usepackage{titlesec}
\usepackage{fancyhdr}
\usepackage{amsmath}

\titleformat{\section}{\large\bfseries}{\thesection}{0.75em}{}
\titleformat{\subsection}{\normalsize\bfseries}{\thesubsection}{0.75em}{}
\titleformat{\subsubsection}{\normalsize\itshape}{\thesubsubsection}{0.75em}{}

\lstset{
  basicstyle=\ttfamily\small,
  breaklines=true,
  frame=single,
  backgroundcolor=\color{gray!5},
  keywordstyle=\color{blue!70!black},
  commentstyle=\color{green!50!black},
  stringstyle=\color{red!60!black}
}

\pagestyle{fancy}
\fancyhf{}
\rhead{StepHabit Capstone}
\lhead{Comprehensive Report}
\cfoot{\thepage}

\begin{document}

\begin{titlepage}
  \centering
  {\Huge StepHabit Capstone Report\\[1em]}
  {\Large A 30-page technical and product deep dive}\\[2em]
  {\large Prepared by: Engineering Team}\\[1em]
  {\large Date: \today}\\[3em]
  \vfill
  {\large Abstract}\\[0.5em]
  StepHabit is a full-stack habit building and productivity platform that combines
  goal tracking, task management, smart scheduling, and AI-assisted coaching. This
  report provides a detailed examination of the system architecture, codebase, and
  implementation decisions, serving as both a design document and a reference for
  future contributors.
  \vfill
\end{titlepage}

\tableofcontents
\newpage

\section{Introduction}
StepHabit is designed to help users build sustainable routines through a blend of
structured scheduling, community accountability, and AI-powered guidance. The
project is implemented as a Node.js and Express backend with a React front end
built on the CoreUI design system. It integrates relational data modeling via
Sequelize, secure authentication flows, and a suite of productivity-oriented
features (habits, tasks, calendars, notifications, achievements, and group
challenges).

This capstone provides the narrative and technical depth typically expected from a
multi-chapter thesis: an overview of motivations, platform architecture,
implementation specifics, data structures, deployment considerations, testing
approaches, and future roadmap. Throughout, the document references concrete code
paths to anchor discussion in the current repository state.

\subsection{Problem Statement and Goals}
Individuals struggle to translate aspirational goals into daily action. StepHabit
aims to solve this by combining three pillars:
\begin{enumerate}
  \item \textbf{Clarity}: capture habits, tasks, and schedules in a single source of
        truth.
  \item \textbf{Momentum}: surface progress, achievements, and reminders to keep
        users engaged.
  \item \textbf{Coaching}: leverage AI prompts and habit plan generation to guide
        users toward actionable steps.
\end{enumerate}

\subsection{Scope of This Report}
The following chapters dissect backend services, data models, REST APIs,
frontend UI flows, and supporting utilities. Each section blends descriptive
text with rationale, patterns, and future-proofing guidance to help maintainers
extend the system with confidence.

\section{System Overview}
\subsection{Technology Stack}
\begin{itemize}
  \item \textbf{Backend}: Node.js with Express and Sequelize ORM for PostgreSQL. Core
        entry point: \texttt{Backend/server.js}.
  \item \textbf{Frontend}: React (Vite) using CoreUI components for rapid admin-style
        layouts and navigation, defined in \texttt{Frontend/src}.
  \item \textbf{AI Services}: LangChain clients configured for Anthropic models to
        support habit plan generation and rewriting.
  \item \textbf{Infrastructure}: Environment-driven configuration via \texttt{dotenv},
        with static uploads served from \texttt{/uploads}.
\end{itemize}

\subsection{High-Level Capabilities}
\begin{itemize}
  \item Habit creation, categorization, daily goals, and progress logging.
  \item Task management with durations, color labels, status transitions, and
        scheduling constraints.
  \item Calendar integration and busy-block tracking to avoid conflicts.
  \item Achievements, group challenges, and community messaging to enhance
        motivation.
  \item Notifications and direct messaging for real-time engagement.
  \item AI habit coaching: plan suggestions and idea rewrites to improve habit
        quality.
\end{itemize}

\section{Backend Architecture}
\subsection{Express Application Entry Point}
The \texttt{server.js} file configures middlewares, route mounts, database sync, and
health checks. On startup, the server authenticates the database connection,
performs cleanup on legacy tables, enforces optional column additions, and then
synchronizes models. It mounts API routers for users, habits, progress, schedules,
notifications, group challenges, achievements, friends, analytics, tasks, avatars,
daily challenges, smart scheduling, library content, AI chat, assistant memories,
calendar data, and messaging.

\subsection{Error Handling Strategy}
A global error handler logs unexpected exceptions and returns a generic HTTP 500
response. Individual controllers validate inputs and return 400 or 404 responses
for user-facing errors (e.g., missing identifiers, invalid statuses).

\subsection{Database Synchronization and Migration Safety}
During startup, the application:
\begin{enumerate}
  \item Authenticates the database connection.
  \item Drops legacy tables (e.g., \texttt{user\_settings}) when found.
  \item Cleans orphaned records across assistant memories, calendar artifacts,
        notifications, progress logs, tasks, habits, and friendships.
  \item Adds optional columns (e.g., \texttt{avatar} on users) if absent.
  \item Executes \texttt{sequelize.sync \{ alter: true \}} to align schemas.
\end{enumerate}
This pattern balances resilience with forward compatibility for incremental
schema evolution.

\section{Data Modeling}
Sequelize models define relational structures for user-centric productivity data.
Associations live in \texttt{Backend/models/index.js}, ensuring a single source of
truth for relationships.

\subsection{Core Entities}
\begin{itemize}
  \item \textbf{User}: owns habits, tasks, calendar events, notifications, settings,
        friendships, achievements, assistant memories, and more.
  \item \textbf{Habit}: tied to a user, with schedules and progress logs to capture
        recurring actions.
  \item \textbf{Task}: user-owned with duration metadata, status, scheduling bounds,
        and aesthetic attributes.
  \item \textbf{Schedule} and \textbf{BusySchedule}: define time blocks for habits and
        calendar conflicts.
  \item \textbf{Progress}: links users to habits with dated completion metrics.
  \item \textbf{Achievement} and \textbf{UserAchievement}: milestone tracking via many-
        to-many relationships.
  \item \textbf{Friend}: self-referential pivot for bidirectional user friendships.
  \item \textbf{GroupChallenge} and \textbf{UserGroupChallenge}: collaborative habit
        competitions and participation tracking.
  \item \textbf{Notification}: user-targeted alerts.
  \item \textbf{AssistantMemory}: AI assistant context tied to users.
  \item \textbf{CalendarIntegration} and \textbf{CalendarEvent}: synchronize external
        calendar data with user accounts.
  \item \textbf{ChatMessage} and \textbf{GroupChallengeMessage}: messaging constructs
        for direct and challenge-specific chats.
  \item \textbf{RegistrationVerification}: support for email verification flows.
\end{itemize}

\subsection{Association Patterns}
Associations emphasize:
\begin{itemize}
  \item Clear ownership (e.g., \texttt{User.hasMany(Habit)}, \texttt{Task.belongsTo(User)}).
  \item Many-to-many bridges for achievements and group challenges via explicit pivot
        tables.
  \item Self-referential friendships through a \texttt{Friend} join model with requester
        and recipient roles.
  \item Hierarchical calendar structures linking integrations to events and users.
\end{itemize}

\section{API Layer}
\subsection{Habit Controller}
The habit controller handles CRUD operations and AI-assisted idea refinement:
\begin{itemize}
  \item \textbf{Listing}: fetches habits by \texttt{user\_id}, ordered by creation time.
  \item \textbf{Creation}: validates required \texttt{title} and \texttt{user\_id}, normalizes
        numeric targets, and persists metadata such as category and daily goals.
  \item \textbf{AI Suggestions}: invokes LangChain-backed services to generate habit
        plans and rewrite habit ideas for clarity.
  \item \textbf{Update/Delete}: supports partial updates, cascaded removal of progress
        and schedules, and consistent 404 responses when records are absent.
\end{itemize}

\subsection{Task Controller}
Task endpoints focus on flexible creation and status management:
\begin{itemize}
  \item \textbf{Listing}: returns tasks for a user ordered by creation date.
  \item \textbf{Creation}: enforces \texttt{name} and \texttt{user\_id}, validates status against
        allowed values, normalizes durations, and stores optional scheduling bounds
        like \texttt{schedule\_after} and \texttt{due\_date}.
  \item \textbf{Status Updates}: ensures valid identifiers and status transitions
        before persisting state changes.
\end{itemize}

\subsection{Route Surface}
The Express server mounts routers for users, habits, progress, schedules,
notifications, group challenges, achievements, friends, analytics, tasks,
avatars, daily challenges, smart scheduler, library, AI chat, assistant,
calendar, messaging, and the generic AI endpoint. Each router encapsulates its
own validation and business logic while sharing authentication and error handling
patterns established in \texttt{server.js}.

\section{AI and Coaching}
AI-powered features rely on LangChain integrations targeting Anthropic models.
The habit controller exposes endpoints to generate structured habit plans and
rewrite freeform habit ideas. These services enable:
\begin{itemize}
  \item \textbf{Behavioral scoping}: breaking ambitious habits into clear steps.
  \item \textbf{Tone and clarity}: rewriting user input into motivating, actionable
        statements.
  \item \textbf{Feedback loops}: embedding AI prompts within the UI (\emph{HabitCoach}) to
        guide users before they commit to new routines.
\end{itemize}

\section{Frontend Architecture}
\subsection{Framework and Layout}
The front end uses React with Vite for tooling and CoreUI for layout primitives.
Navigation is defined in \texttt{src/\_nav.js}, grouping routes into \emph{Plan \\& track},
\emph{Connect}, and \emph{You} sections. The base layout provides a sidebar, header,
and content area consistent with admin dashboards.

\subsection{Routing and Lazy Loading}
Routes in \texttt{src/routes.js} lazy-load feature bundles for authentication,
dashboards, habits, planner, notifications, tasks, profile, community, and support
pages. This keeps the initial bundle lean while supporting a wide surface of
productivity tools.

\subsection{Key Screens}
\begin{itemize}
  \item \textbf{Dashboard}: centralizes user metrics, reminders, and navigation
        shortcuts.
  \item \textbf{Planner and Schedules}: offer calendar-style overviews and smart
        scheduling options.
  \item \textbf{Tasks and Habits}: dedicated pages for CRUD operations, progress
        visualization, and AI habit coaching.
  \item \textbf{Community}: friends, messaging, group challenges, and leaderboards
        to foster accountability.
  \item \textbf{Profile}: personal settings, avatars, and notification preferences.
  \item \textbf{Support}: contact forms and help center resources.
\end{itemize}

\section{Security and Validation}
\subsection{Authentication and Authorization}
JWT-based authentication (via \texttt{jsonwebtoken}) secures protected endpoints.
Password storage leverages \texttt{bcryptjs}. Middleware layers would typically
verify tokens before reaching controller logic (implementation details reside in
route modules beyond this report's excerpt).

\subsection{Input Validation and Error Semantics}
Controllers validate required fields (e.g., \texttt{user\_id}, \texttt{name}) and acceptable status
values. Responses consistently use 400 for malformed requests and 404 for missing
resources, while the global error handler standardizes 500 responses for
unexpected failures.

\subsection{Data Integrity}
Startup routines proactively remove orphaned records before syncing schemas,
reducing referential drift. Associations in \texttt{models/index.js} enforce
foreign-key relationships, and cleanup helpers delete dependent records when
parents are removed (e.g., habit deletion cascades to progress and schedules).

\section{Scheduling and Productivity Workflows}
\subsection{Habits vs. Tasks}
Habits capture recurring behaviors with optional daily goals and categories,
whereas tasks represent discrete units of work with durations and due dates.
Both types surface on planners and dashboards, but habits emphasize streaks and
progress logs, while tasks emphasize completion status and time allocation.

\subsection{Smart Scheduling and Busy Blocks}
Busy schedules protect time on the calendar, preventing overbooking. Planner
routes connect to smart scheduling logic (backend service not shown in this
excerpt) to suggest optimal time slots based on availability and task metadata.

\subsection{Progress and Achievements}
Progress entries timestamp habit completions. Achievements and group challenges
translate those completions into motivational milestones, with many-to-many
links enabling users to share progress and compete collaboratively.

\section{Community and Communication}
\subsection{Friendships and Group Challenges}
Friend relationships are modeled through a self-referential join table, allowing
bidirectional invites and confirmations. Group challenges link users to
collective goals, with messages scoped to challenges for contextual discussion.

\subsection{Notifications and Messaging}
Notifications provide user-specific alerts (e.g., reminders, challenge updates),
while direct messages enable one-to-one communication. The chat model tracks
senders and recipients, preserving conversational history.

\section{Deployment Considerations}
\subsection{Environment Management}
The backend expects environment variables for database connectivity, JWT secrets,
and third-party integrations. The frontend can be configured via Vite's
environment handling for API base URLs and feature toggles.

\subsection{Build and Runtime}
\begin{itemize}
  \item \textbf{Backend}: run \texttt{npm install} then \texttt{npm run start} (or \texttt{npm run
        dev} with nodemon) in \texttt{Backend/}.
  \item \textbf{Frontend}: run \texttt{npm install} then \texttt{npm run dev} in \texttt{Frontend/},
        accessing the app via the configured host and port.
\end{itemize}

\subsection{Static Assets and Uploads}
The Express server serves uploaded files from \texttt{/uploads}, enabling avatar and
attachment features without requiring a separate CDN in development.

\section{Testing and Quality}
\subsection{Manual and Automated Checks}
While automated test suites are not included in the current repository snapshot,
quality assurance can follow these pillars:
\begin{itemize}
  \item Unit tests for controllers to validate HTTP semantics and error handling.
  \item Integration tests exercising Sequelize models and associations.
  \item End-to-end UI tests covering critical flows: registration, login, habit
        creation, task completion, and AI coaching interactions.
\end{itemize}

\subsection{Static Analysis}
Linters (ESLint) and formatters (Prettier) can be integrated in both frontend and
backend pipelines. Type checking with TypeScript or JSDoc annotations would
further reduce runtime defects.

\section{Future Work}
\begin{itemize}
  \item \textbf{Advanced Analytics}: richer habit adherence trends, burn-down charts
        for tasks, and predictive scheduling.
  \item \textbf{Collaboration}: shared planners for teams or families, with role-based
        permissions.
  \item \textbf{Mobile Experience}: React Native client reusing API contracts to reach
        more platforms.
  \item \textbf{Offline Support}: local-first sync for tasks and habits, with conflict
        resolution strategies.
  \item \textbf{Extensible AI}: personalized coaching tuned to user progress and
        contextual data.
\end{itemize}

\section{Conclusion}
StepHabit blends goal tracking, scheduling, social accountability, and AI
assistance into a cohesive platform. Its architecture balances modularity (via
Express routers and Sequelize models) with rapid UI development (through CoreUI
components and lazy-loaded routes). The cleanup routines and association
mappings provide a strong foundation for data integrity, while the AI endpoints
introduce differentiated value for habit formation. This report captures the
current blueprint and offers guidance for continued iteration toward a more
capable and supportive productivity companion.

\end{document}
